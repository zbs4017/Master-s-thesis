\chapter{绪\hspace{6pt}论}

随着硬件技术的快速发展,以RDMA分离式内存、持久性内存和CXL为代表的新型技术为优化内存资源利用率提供了全新思路。这些技术能够支持将访问频率较低的冷数据迁移至低成本存储设备,并按实际需求动态换入内存,有效缓解了内存容量压力。然而,现有研究主要关注通过应用程序接口(API)实现冷数据迁移的显式管理达到最优性能,这导致未经专门优化的传统应用程序难以直接从中受益。基于Linux内核缺页中断机制的透明冷数据卸载技术为解决这一问题提供了新的途径。这种方案无需修改应用程序即可自动实现冷热数据的识别与管理,但现有研究较少涉及如何在异构存储后端环境下准确估算工作集这一关键问题。本研究致力于填补这一研究空白,通过探索面向异构存储后端的高效工作集估计算法以及冷热页面识别算法,以期在容器化环境下显著提升内存资源利用效率,实现内存成本的优化。

\section{研究工作的背景与意义}

近年来,随着互联网技术的迅猛发展和数据密集型应用的广泛普及,工作负载对内存容量的需求呈指数级增长。与此同时,以容器化为代表的虚拟化部署技术日趋成熟并在业界得到广泛应用,进一步加剧了内存资源的紧张程度,使得内存虚拟化开销问题愈发凸显。尽管现代单机服务器的内存容量已提升至TB级别,但其成本亦随之大幅攀升。Patel等人\citing{patel2023xfm}的研究指出,内存成本已占据数据中心总成本的50\%以上,这不仅给数据中心运营商带来了沉重的经济负担,也与当前倡导绿色节能的可持续发展理念相悖。

然而,与高昂的内存成本形成鲜明对比的是,实际应用中内存资源的利用率却普遍偏低。这种“高成本,低利用”的现象已成为制约数据中心资源高效利用的关键瓶颈。Reiss等人\citing{reiss2012heterogeneity}对生产集群的内存利用率进行了统计分析,结果表明集群内的内存利用率存在严重的不平衡性:在70\%的运行时间内,平均有30\%的内存处于空闲状态;而在被分配的80\%内存中,平均只有50\%被实际使用。造成这种现象的原因是多方面的。首先,现代数据中心通常需要同时支持多种类型的工作负载,包括长期运行的在线服务、批处理任务、数据分析任务等,这些负载在资源需求和运行时长上存在显著差异,且集群中的机器配置也并非同构,这种工作负载和硬件资源的异构性导致了难以进行统一高效的资源调度。其次,为了保证服务质量(Service Level Objective, SLO),系统必须按照峰值需求配置资源,这虽然会导致非高峰期出现资源闲置,但这种冗余对于应对流量突增、保障服务稳定性是必要的。第三,负载具有明显的动态特征,例如,许多应用的访问请求量在日间和夜间存在显著差异,这种周期性的变化要求系统预留足够的资源来处理高峰期需求,从而导致低谷时段的资源利用率下降。最后,任务执行的固有特征也会影响内存利用率,例如,长期运行的任务通常会预留更多内存作为缓冲以优化性能,同时JVM运行时环境和系统级别的内存管理机制(如页缓存)也会占用一定的内存空间,这些因素共同作用导致了内存资源的利用率难以得到有效提升。

将冷数据迁移至低成本存储设备,构建异构分层内存系统,理论上可以在保证应用性能的同时显著降低总体拥有成本 (TCO)。然而,在早期计算机系统中,由于磁盘的访问延迟过高,频繁的内存换入换出操作会导致系统性能的急剧下降,因此该策略并未得到广泛应用。当时的研究和实践更多地致力于避免而非利用页面交换机制。

近年来,新型非易失性存储技术(如NVMe SSD、PMem)以及基于RDMA的高速远程内存访问技术的兴起,使得异构分层存储设备的访问延迟显著降低,为构建高效的分层内存系统提供了新的契机。这些技术进步促使工业界和学术界重新审视并积极探索如何利用这些新型存储技术构建高效的分层内存系统,以提升内存资源利用率。RDMA技术凭借其出色的网络延迟和带宽性能,以及支持直接访问远程设备内存的特性,使得将冷内存页迁移至远端内存池成为可能。持久性内存不仅具备数据持久化能力,还拥有超大容量和接近DRAM的访问延迟,因此可以作为理想的冷数据承载介质。此外,CXL作为一种新兴的高速互联技术,也因其卓越的性能潜力而受到业界的广泛关注。

上述新型存储和互联技术均支持将冷内存页进行卸载。在容器化部署环境下,若能准确估算并实时监控容器中应用负载的工作集大小,并结合Linux内核的内存回收机制,主动地将冷内存页透明地卸载至异构存储设备,则有望在保障服务质量(QoS)的前提下,显著提升单台物理服务器可承载的容器数量,从而进一步提高数据中心的资源利用率和经济效益。



\section{国内外研究历史与现状}

\subsection{分层内存研究历史与现状}

早期操作系统提出的虚拟内存的目的是为了解决内存容量不足的问题,通过换入换出机制,来实现内存的扩容。

之后unix的分页机制,本意是为了减少内存碎片和内存浪费。之后在这个功能上进行了拓展,演进出以页为单位的换入换出技术,这是分层内存的雏形。

这个时候后端存储设备还是软盘为主,这个技术主要是为了防止应用在运行过程中,内存不足,导致应用崩溃。

之后存储设备不断发展,磁盘,ssd这些,但是性能还是不够。分层内存技术没有得到广泛应用,并且时候,避免数据的频繁换入换出成为共识。

之后,硬件设备的发展,使得分层内存技术得到了新的发展。不同的延迟和带宽,让存储金字塔变得比较丰富。

基于rdma的远程内存,他敏锐的发现了,不同节点的内存负载不均衡,可以将自己的冷数据迁移到其他节点,从而实现内存的利用率的提高。

非易失性内存内存也是一个重要的发展,他的容量有了很大的提升,可以轻松的拓展到TB级别,并且有接近DRAM的访问延迟,可以作为冷数据存储的介质,可以说是分层内存的理想介质。学术界针对这一硬件做了大量拓展,

然后近些年来,cxl这种互联技术,可以轻易的拓展,放我那儿ssd这些设备有了更低的设备,可以作为冷数据存储的介质。

压缩内存也可以作为一种冷内存存储,不过他是软件实现的。他通过将内存数据压缩,从而减少内存的占用。之后也可以解压,这种技术的性能主要取决于压缩比。

可以看到,硬件的发展,为分层内存技术提供了新的发展契机。

\subsection{工作集估计算法研究历史与现状}
\section{本文的主要贡献与创新}


\section{本论文的结构安排}
本文的章节结构安排如下:

\footnote{脚注序号“\ding{172},……,\ding{180}”的字体是“正文”,不是“上标”,序号与脚注内容文字之间空1个半角字符,脚注的段落格式为:单倍行距,段前空0磅,段后空0磅,悬挂缩进1.5字符;中文用宋体,字号为小五号,英文和数字用Times New Roman字体,字号为9磅;中英文混排时,所有标点符号(例如逗号“,”、括号“()”等)一律使用中文输入状态下的标点符号,但小数点采用英文状态下的样式“.”。}