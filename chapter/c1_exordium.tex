\chapter{绪\hspace{6pt}论}

随着硬件技术的快速发展,以RDMA分离式内存、持久性内存和CXL为代表的新型技术为优化内存资源利用率提供了全新思路。这些技术能够支持将访问频率较低的冷数据迁移至低成本存储设备,并按实际需求动态换入内存,有效缓解了内存容量压力。然而,现有研究主要关注通过应用程序接口(API)实现冷数据迁移的显式管理达到最优性能,这导致未经专门优化的传统应用程序难以直接从中受益。基于Linux内核缺页中断机制的透明冷数据卸载技术为解决这一问题提供了新的途径。这种方案无需修改应用程序即可自动实现冷热数据的识别与管理,但现有研究较少涉及如何在异构存储后端环境下准确估算工作集这一关键问题。本研究致力于填补这一研究空白,通过探索面向异构存储后端的高效工作集估计算法以及冷热页面识别算法,以期在容器化环境下显著提升内存资源利用效率,实现内存成本的优化。

\section{研究工作的背景与意义}

近年来,内存技术发展与应用需求之间的矛盾日益凸显。在技术层面,DRAM制造工艺正面临严峻挑战。如Onur Mutlu\citing{mutlu2013memory}所指出,随着工艺节点向10nm及以下推进,DRAM存储单元的微缩已接近物理极限,位密度提升显著放缓。为维持性能增长,业界不得不采用多重曝光(Multiple Patterning)和极紫外光刻(EUV Lithography)等复杂工艺,导致制造成本大幅上升。Patel等人\citing{patel2023xfm}的研究表明,当前内存成本已占据数据中心总运营成本的50\%以上,制约了数据中心的发展。同时,DRAM的刷新功耗随容量增长持续攀升,进一步加剧了能源消耗问题。

与DRAM技术发展受限形成鲜明对比的是,现代互联网应用、容器技术以及大数据分析、人工智能等新兴领域的数据密集度急剧增加,对内存容量、带宽、延迟和吞吐量提出了前所未有的高要求。研究表明,典型数据中心的内存需求正以每年约40\%的速度增长。这种供需失衡使得内存资源的高效利用变得尤为重要。

然而,实际生产环境中的内存利用状况却不容乐观。Reiss等人\citing{reiss2012heterogeneity}对大规模生产集群的统计分析显示,在70\%的运行时间内,集群平均有30\%的内存处于闲置状态;而在已分配的内存中,实际使用率仅为50\%。这种"高成本,低利用"的现象主要由以下因素导致:首先,现代数据中心需要同时支持在线服务、批处理任务、数据分析等多种类型的工作负载,其资源需求和运行时长存在显著差异,导致资源分配难以优化;其次,为保证服务质量(Service Level Objective, SLO),系统通常需要按照峰值需求预留内存资源,造成非高峰期的资源闲置;最后,负载的周期性特征(如日间与夜间的访问量差异)以及任务执行的固有特性(如JVM运行时环境的内存开销)进一步加剧了内存利用率低下的问题。

新型存储和互联技术的快速发展为解决内存系统面临的挑战提供了多种的解决方案。NVMe SSD、持久性内存(Persistent Memory, PMem)等新型存储设备显著降低了存储访问延迟,而基于RDMA的高速网络技术实现了高效的远程内存访问。同时,基于软件的内存压缩技术的成熟为提升内存密度提供了新的技术路径,这些进步为构建异构分层内存系统(Heterogeneous Memory Hierarchy)奠定了技术基础。具体而言,RDMA技术凭借其微秒级的访问延迟和零拷贝特性,使得远程内存访问成为可能;持久性内存则因其接近DRAM的访问速度和字节寻址能力,成为理想的冷数据存储介质。此外,内存压缩技术通过对内存数据进行实时压缩和解压缩,在保证访问性能的同时显著提升了内存的有效容量。

基于上述技术进展,通过将冷数据迁移至低成本存储设备,构建异构分层内存系统,理论上可以在保证应用性能的同时显著降低总体拥有成本(TCO)。在早期计算机系统中,由于磁盘的访问延迟过高,频繁的内存换入换出操作会导致系统性能的急剧下降,因此该策略并未得到广泛应用。然而,随着新型存储和互联技术的成熟,这一方案重新成为可能。

在容器化部署环境下,通过准确估算并实时监控容器中应用负载的活跃工作集大小,并结合Linux内核的内存回收机制,主动地将冷内存页透明地卸载至异构存储设备,有望在保障服务质量(QoS)的前提下,显著提升单台物理服务器可承载的容器数量,从而进一步提高数据中心的资源利用率和经济效益。这一研究不仅具有重要的理论价值,也为解决当前内存资源利用效率低下的实际问题提供了可行的技术路径。

\section{国内外研究历史与现状}

\subsection{分层内存研究历史与现状}

早期操作系统提出的虚拟内存的目的是为了解决内存容量不足的问题,通过换入换出机制,来实现内存的扩容。

之后unix的分页机制,本意是为了减少内存碎片和内存浪费。之后在这个功能上进行了拓展,演进出以页为单位的换入换出技术,这是分层内存的雏形。

这个时候后端存储设备还是软盘为主,这个技术主要是为了防止应用在运行过程中,内存不足,导致应用崩溃。

之后存储设备不断发展,磁盘,ssd这些,但是性能还是不够。分层内存技术没有得到广泛应用,并且时候,避免数据的频繁换入换出成为共识。

之后,硬件设备的发展,使得分层内存技术得到了新的发展。不同的延迟和带宽,让存储金字塔变得比较丰富。

基于rdma的远程内存,他敏锐的发现了,不同节点的内存负载不均衡,可以将自己的冷数据迁移到其他节点,从而实现内存的利用率的提高。

非易失性内存内存也是一个重要的发展,他的容量有了很大的提升,可以轻松的拓展到TB级别,并且有接近DRAM的访问延迟,可以作为冷数据存储的介质,可以说是分层内存的理想介质。学术界针对这一硬件做了大量拓展,

然后近些年来,cxl这种互联技术,可以轻易的拓展,放我那儿ssd这些设备有了更低的设备,可以作为冷数据存储的介质。

压缩内存也可以作为一种冷内存存储,不过他是软件实现的。他通过将内存数据压缩,从而减少内存的占用。之后也可以解压,这种技术的性能主要取决于压缩比。

可以看到,硬件的发展,为分层内存技术提供了新的发展契机。

\subsection{工作集估计算法研究历史与现状}
\section{本文的主要贡献与创新}


\section{本论文的结构安排}
本文的章节结构安排如下:

\footnote{脚注序号“\ding{172},……,\ding{180}”的字体是“正文”,不是“上标”,序号与脚注内容文字之间空1个半角字符,脚注的段落格式为:单倍行距,段前空0磅,段后空0磅,悬挂缩进1.5字符;中文用宋体,字号为小五号,英文和数字用Times New Roman字体,字号为9磅;中英文混排时,所有标点符号(例如逗号“,”、括号“()”等)一律使用中文输入状态下的标点符号,但小数点采用英文状态下的样式“.”。}