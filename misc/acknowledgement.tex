
\thesisacknowledgement

三年时光如白驹过隙,研究生生涯即将画上句号。在这篇论文完成之际,心中充满感激之情,谨以此文向所有在求学道路上给予帮助、关怀与支持的师长、同学和亲人表达最诚挚的谢意。

首先,由衷感谢导师段翰聪教授的悉心指导与无私付出。段老师严谨的治学态度、渊博的学识以及精益求精的科研精神,为学术研究指明了方向。在论文选题、实验设计以及写作过程中,段老师始终耐心指导,提出了许多宝贵的建议,使研究得以顺利进行。不仅如此,段老师为人谦和、待人真诚,其高尚的品格与敬业精神深深感染了我,成为今后工作与学习的榜样。

其次,感谢NDSL实验室的全体老师。在课程学习与科研实践中,老师们以深厚的学术造诣和丰富的实践经验,为我的专业能力提升提供了重要帮助。特别感谢在论文开题、中期检查以及答辩准备阶段给予的宝贵意见,这些建议使论文不断完善,研究思路更加清晰。老师们严谨的学术态度与诲人不倦的精神,让我受益匪浅。

同时,衷心感谢NDSL实验室的同学们。在共同学习与科研的日子里,大家相互支持、共同进步,建立了深厚的友谊。无论是学术问题的探讨,还是生活中的点滴帮助,都让我感受到集体的温暖与力量。特别感谢在实验过程中给予的技术支持与建议,以及在论文写作期间提供的宝贵意见。这段同窗之情将成为研究生生涯中最珍贵的回忆。

此外,还要感谢学校提供的优质学习资源与科研平台。图书馆丰富的文献资料、实验室先进的仪器设备,为论文的顺利完成提供了坚实的保障。同时,感谢研究生院以及学院各位老师在学业管理与生活支持方面的辛勤付出,为我们的学习与科研创造了良好的环境。

最后,向我的父母和家人致以最深切的感谢。三年来,他们始终默默支持我的学业,给予无条件的理解与鼓励。无论遇到怎样的困难与挑战,家人的关怀与陪伴始终是我前行的动力。他们的爱与付出,让我能够心无旁骛地专注于学业,顺利完成研究生阶段的学习任务。

研究生生涯的结束,是人生新阶段的开始。这段经历不仅让我收获了知识与能力,更让我懂得了感恩与责任。在此,向所有关心、帮助过我的人致以最诚挚的谢意,愿你们健康平安,万事顺遂!

谨以此文,献给所有在求学道路上给予我支持与鼓励的人。