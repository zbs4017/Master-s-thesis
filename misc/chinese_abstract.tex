
\begin{chineseabstract}

近年来,机器学习、图计算和内存数据库等数据密集型应用迅猛发展,加之容器化部署的虚拟化开销,导致系统内存需求急剧增长。然而,传统DRAM技术因物理缩放限制和市场波动面临成本持续攀升的挑战,而应用程序内存利用率普遍偏低,存在大量冷页面。同时,NVMe SSD、非易失性存储器(Non-Volatile Memory, NVM)、快速计算链接(Compute Express Link, CXL)和基于远程直接内存访问(Remote Direct Memory Access, RDMA)的远程内存等新型存储与互连技术,以及软件内存压缩技术的发展,为内存系统提供了更高容量、更低成本与功耗的新选择。

在此背景下,主动将应用程序中的冷页面卸载至异构后端已成为一种极具潜力的成本优化策略。通过将异构卸载后端接入Linux内存回收机制,可实现对用户程序完全透明的卸载过程。然而,传统Linux内存管理存在两个主要局限:其回收策略偏向文件页,损害文件密集型应用性能;采用被动响应式回收策略,仅在内存紧张时触发,无法充分利用高性能异构后端进行冷页面主动卸载。

为了解决上述传统Linux内存管理的局限性并充分发挥异构存储技术的优势,本研究提出了面向异构后端的冷页面自适应主动卸载方案。该方案主要解决两个关键问题:针对不同异构卸载后端确定适当的卸载规模;精确识别系统中的冷热页面。本方案的核心创新点如下:
\begin{itemize}
    \item 设计了基于内存压力的自适应主动卸载机制,能针对不同异构后端自适应地确定卸载规模。该机制以内存同步回收延迟时间作为压力指标,实现在多样化应用场景下的智能卸载调整,并通过与Linux CGroup框架集成,精确控制应用内存的主动卸载。
    \item 提出了一种基于重用距离(页面两次连续访问之间的页面数量)的冷热页识别算法。针对现有页面冷热识别算法倾向于回收文件页,对文件密集型应用性能不利的问题,本算法基于页面重用距离,能够更准确地识别文件热页,从而提升文件密集型应用的性能。
\end{itemize}

本研究开发了相应的原型系统,并针对容器化应用场景进行了实验验证。实验结果表明,该系统能够有效识别不同异构后端存储架构下的卸载规模,在确保服务性能的同时,与基线系统相比,实现容器内存使用量5\%\~{}14\%的显著降低。

\chinesekeyword{分层内存,重用距离,冷热页面识别}
\end{chineseabstract}