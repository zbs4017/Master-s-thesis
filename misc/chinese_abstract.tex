
\begin{chineseabstract}

近年来,机器学习、图计算与内存数据库等数据密集型工作负载,以及容器化部署带来的虚拟化开销,共同推高系统对大容量、高带宽内存的需求;然而传统 DRAM 受物理缩放瓶颈和市场波动影响,单位容量成本下降放缓且成本攀升,而应用程序实际内存利用率往往不足,产生大量冷页面。针对这一矛盾,产业与学术界提出多元化替代路径:在介质层面,基于NVMe协议的固态硬盘(Non-Volatile Memory Express Solid State Drive,NVMe SSD)与非易失性内存(Non-Volatile Memory,NVM)供给高密度、低成本存储;在互连层面,快速计算链接(Compute Express Link,CXL)与基于远程直接内存访问(Remote Direct Memory Access,RDMA)的远程内存扩展带宽与访问弹性;在软件层面,内存压缩技术缓解容量压力。借助这些技术,主动将冷页面卸载至异构后端成为有效的成本优化策略;把异构后端纳入 Linux 内存回收机制,可在应用透明前提下持续迁移冷数据。然而,现有 Linux 内存管理仍面临两大局限:其回收逻辑偏向文件页,易损害文件密集型应用性能;且采用被动、压力驱动模型,仅在内存紧张时触发,难以充分利用高性能异构后端实现冷页面的主动、持续卸载。

为克服传统 Linux 内存管理的局限,本研究提出面向异构后端的冷页面自适应主动卸载方案,重点解决两大问题:其一,为不同卸载后端设定合适的卸载规模;其二,精准识别系统冷热页面。核心创新点包括:

\begin{itemize}
    \item 设计了基于内存压力的自适应主动卸载机制,能针对不同异构后端自适应地确定卸载规模。该机制以内存同步回收延迟时间作为压力指标,实现在多样化应用场景下的自适应卸载调整,并通过与Linux 控制组(Control Group, CGroup)框架集成,精确控制应用内存的主动卸载。
    \item 提出了一种基于重用距离(页面两次连续访问之间的页面数量)的冷热页识别算法。针对现有页面冷热识别算法倾向于回收文件页,对文件密集型应用性能不利的问题,本算法基于页面重用距离,能够更准确地识别文件热页,从而提升文件密集型应用的性能。
\end{itemize}

本研究开发了相应的原型系统,并针对容器化应用场景进行了实验验证。实验结果表明,该系统能够有效识别不同异构后端存储架构下的卸载规模,在确保服务性能的同时,与基线系统相比,实现容器内存使用量5\%\~{}14\%的显著降低。

\chinesekeyword{分层内存,重用距离,冷热页面识别}
\end{chineseabstract}