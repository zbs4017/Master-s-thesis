	
\begin{chineseabstract}

随着数据爆炸式增长和应用计算需求日趋复杂,数据中心内存资源愈发重要。容器化虽已成为主流部署范式,但其虚拟化开销加剧了内存需求。尽管单机内存容量提升,但在大规模容器集群中,内存成本仍是数据中心 TCO 的主要组成部分。因此,亟需针对容器环境进行内存优化,探索更具成本效益的解决方案。

分层内存架构通过迁移冷数据至低成本介质,为内存资源的高效利用提供了新思路。然而,NVMe SSD、NVM、RDMA 内存池及压缩内存等多样化后端存储方案的涌现,因其各异的延迟和带宽特性,使得传统工作集估计算法难以跨平台适用,而为每种后端定制算法又缺乏可扩展性,且难以适应 CXL 内存池等新技术。

针对异构存储环境下的内存管理挑战,本文提出了一种内核-用户态协同的自适应工作集计算方法。该方法在内核态监测并量化内存压力,反馈给用户态以指导其动态调整工作集大小,并通过 cgroup 机制进行内存限制。结合改进的冷热页识别算法,内核可将冷页迁移至低成本存储,优化内存利用率,从而实现针对不同后端存储的自适应内存管理。

本文的主要工作内容和创新点如下:

1. 提出了一种新颖的基于延迟的内存压力感知架构。该架构以内存回收引起的程序延迟作为内存压力的表征,通过监测内存回收所占用的CPU时间来量化内存压力。该方法能够针对不同后端存储的性能特征进行差异化的内存压力评估,并将该信息暴露给用户态程序.

2. 提出了一种基于内存压力的最小工作集识别算法。该算法根据实时监测的内存压力动态计算容器的最小工作集大小,并利用cgroup机制限制容器的内存使用,触发内存回收,进而实现内存资源的有效节约。

3. 提出了一种基于访问距离的文件热页识别方法。针对现有页面冷热识别算法倾向于回收文件页、对文件密集型应用(如数据库和存储系统)不友好的问题,本方法基于页面访问距离,能够更准确地识别文件热页,从而提升文件密集型应用的性能。

实验结果表明,本文提出的系统可以针对不同的异构后端存储,准确识别容器的工作集,有效减少容器的内存使用量(5\%-14\%),并降低虚拟化开销。

\chinesekeyword{分层内存,工作集大小,内存管理,冷热页面识别}
\end{chineseabstract}