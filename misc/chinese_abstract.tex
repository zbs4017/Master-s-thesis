	
\begin{chineseabstract}

近年来,随着机器学习、图计算和云计算等数据密集型应用的迅猛发展,系统内存容量需求急剧增长,而容器化部署所引入的虚拟化开销进一步加剧了此需求。然而,传统DRAM技术面临物理缩放限制和市场价格波动等挑战,导致内存成本持续攀升。与此同时,应用程序的内存利用率普遍偏低,存在大量冷内存。另一方面,NVMe SSD、新型非易失性存储器(Non-Volatile Memory, NVM)、计算快速链接(Compute Express Link, CXL)设备以及基于RDMA的远程内存访问等新型存储与互连技术迅速发展,为内存系统架构带来了新的机遇。此外,基于软件的内存压缩技术亦不断进步。这些新兴技术在提供更高存储容量的同时,显著降低了单位容量的成本与功耗。

在此背景下,主动将应用程序中的冷内存卸载至异构后端存储,已成为一种极具潜力的成本优化策略。该策略的核心在于:利用内核的Frontswap接口实现对异构后端存储的接入,并与内核既有的换入换出机制无缝集成;为实现主动卸载并精细化管理,通过集成CGroup机制,实现主动卸载并精确控制内存卸载规模。关键在于,整个卸载过程对用户程序完全透明,无需对应用程序进行任何修改即可实现。

针对这些问题,本文的主要工作内容和创新点如下:
\begin{itemize}
    \item 提出了一种基于内存压力的自适应主动卸载方案。该方案能够自适应不同的卸载后端性能特性与用户负载状况。基于量化的内存压力,系统能够智能决策主动卸载的内存规模。通过与CGroup内存回收机制的紧密结合,实现了对应用程序内存的精准、主动卸载。
    \item 提出了一种基于访问距离的冷热页识别算法。针对现有页面冷热识别算法倾向于回收文件页、对文件密集型应用不友好的问题,本方法基于页面访问距离,能够更准确地识别文件热页,从而提升文件密集型应用的性能。
\end{itemize}

本研究开发了相应的原型系统,并针对容器化Web应用场景进行了实验验证。实验结果表明,该系统能够有效识别不同异构后端存储架构下的容器工作集,在确保服务性能的同时,实现容器内存使用量5\%-14\%的显著降低。

\chinesekeyword{分层内存,工作集大小,内存管理,冷热页面识别}
\end{chineseabstract}