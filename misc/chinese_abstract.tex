	
\begin{chineseabstract}

近年来,随着机器学习、图计算和云计算等数据密集型应用的迅猛发展,系统内存容量需求急剧增长,而容器化部署所引入的虚拟化开销进一步加剧了内存需求。然而,传统DRAM技术面临物理缩放限制和市场价格波动等挑战,导致内存成本持续攀升。与此同时,应用程序的内存利用率普遍偏低,存在大量冷页面。另一方面,NVMe SSD、新型非易失性存储器(Non-Volatile Memory, NVM)、计算快速链接(Compute Express Link, CXL)设备以及基于RDMA的远程内存访问等新型存储与互连技术迅速发展,为内存系统架构带来了新的机遇。此外,基于软件的内存压缩技术也不断进步。这些新兴技术在提供更高存储容量的同时,显著降低了单位容量的成本与功耗。

在此背景下,主动将应用程序中的冷页面卸载至异构后端已成为一种极具潜力的成本优化策略。一种实现方式是将异构卸载后端接入Linux中的内存回收机制,使整个卸载过程对用户程序完全透明,无需对应用程序进行任何修改。尽管高性能的异构后端能够将卸载对应用性能的影响降至最低,但该策略仍面临两个关键挑战:确定最优卸载规模及精准识别冷页面。现有研究往往局限于特定卸载后端或应用场景的优化,缺乏自适应能力,难以在不同卸载后端和应用场景间实现灵活适配。

针对这些问题,本文的主要工作内容和创新点如下:
\begin{itemize}
    \item 提出了一种基于内存压力的自适应主动卸载方案。该方案以内存同步回收占用的时间作为压力度量基础,能够在异构后端与不同应用场景下,自适应地决定卸载规模。结合 CGroup 的资源回收机制,实现对应用程序内存的主动卸载。
    \item 提出了一种基于重用距离(页面两次连续访问之间的页面数量)的冷热页识别算法。针对现有页面冷热识别算法倾向于回收文件页,对文件密集型应用不友好的问题,本算法基于页面重用距离,能够更准确地识别文件热页,从而提升文件密集型应用的性能。
\end{itemize}

本研究开发了相应的原型系统,并针对容器化应用场景进行了实验验证。实验结果表明,该系统能够有效识别不同异构后端存储架构下的卸载规模,在确保服务性能的同时,与基线系统相比,实现容器内存使用量5\%\~{}14\%的显著降低。

\chinesekeyword{分层内存,工作集大小,重用距离,冷热页面识别}
\end{chineseabstract}