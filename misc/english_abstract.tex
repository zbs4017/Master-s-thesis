\begin{englishabstract}

In recent years, the continuous development of data-intensive applications such as machine learning, graph computing, and in-memory databases, coupled with the virtualization overhead of containerized deployment, has significantly increased system memory demands. However, traditional DRAM technology faces limitations in physical scaling and is affected by market fluctuations, leading to continuously rising costs, while application memory utilization is generally low, with a large number of cold pages. Simultaneously, emerging storage and interconnect technologies such as Non-Volatile Memory Express Solid State Drives (NVMe SSDs), Non-Volatile Memory (NVM), Compute Express Link (CXL), and RDMA (Remote Direct Memory Access)-based remote memory, as well as software memory compression techniques, provide alternative solutions with higher capacity, lower cost, and reduced power consumption for memory systems.

In this context, proactively offloading cold pages from applications to heterogeneous backends is an effective cost optimization strategy. By integrating heterogeneous offloading backends into the Linux memory reclaim mechanism, the offloading process can be made fully transparent to user programs. However, traditional Linux memory management suffers from two limitations: its reclaim policy favors file pages, negatively impacting the performance of file-intensive applications; it employs a passive reactive reclaim strategy, triggered only under memory pressure, making it difficult to leverage high-performance heterogeneous backends for proactive cold page offloading.

To address the limitations of traditional Linux memory management and fully exploit the advantages of heterogeneous storage technologies, this study proposes an adaptive proactive cold page offloading scheme for heterogeneous backends. This scheme primarily tackles two key challenges: determining the appropriate offloading scale for different heterogeneous backends; and accurately identifying cold and hot pages in the system. The core innovations of this research include:
\begin{itemize}
    \item Designing an adaptive proactive offloading mechanism based on memory pressure, which can adaptively determine the offloading scale for different heterogeneous backends. This mechanism uses the synchronous reclaim latency as a pressure indicator to achieve adaptive offloading adjustments in diverse application scenarios and, through integration with the Linux Control Group (CGroup) framework, precisely controls the proactive offloading of application memory.
    \item Proposing a cold and hot page identification algorithm based on reuse distance (the number of distinct pages accessed between two consecutive accesses to a page). Addressing the issue that existing page coldness/hotness identification algorithms tend to reclaim file pages, which is detrimental to the performance of file-intensive applications, this algorithm, based on page reuse distance, can more accurately identify hot file pages, thereby improving the performance of file-intensive applications.
\end{itemize}

    This research has developed a corresponding prototype system and conducted experimental validation in containerized application scenarios. The experimental results demonstrate that the system can effectively identify the offloading scale for different heterogeneous backend storage architectures, and while ensuring service performance, it achieves a significant reduction in container memory usage by 5\% to 14\% compared to baseline systems.


\englishkeyword{Tiered Memory, Reuse Distance, Hot-Cold Page Identification}

\end{englishabstract}


