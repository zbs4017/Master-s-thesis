\begin{englishabstract}

In recent years, data-intensive applications such as machine learning, graph computing, and in-memory databases have developed rapidly, while the virtualization overhead of containerized deployments has led to a dramatic increase in system memory demands. However, traditional DRAM technology faces challenges of continuously rising costs due to physical scaling limitations and market fluctuations, while application memory utilization remains generally low with numerous cold pages. Meanwhile, emerging storage and interconnection technologies—including NVMe SSDs, Non-Volatile Memory (NVM), Compute Express Link (CXL), and RDMA (Remote Direct Memory Access)-based remote memory access—along with software memory compression techniques, offer new options for memory systems with higher capacity and lower cost and power consumption.

Against this background, actively offloading cold pages from applications to heterogeneous backends has become a promising cost optimization strategy. By integrating heterogeneous offloading backends into the Linux memory reclamation mechanism, the offloading process can be made completely transparent to user programs. However, traditional Linux memory management has two major limitations: its reclamation strategy favors file pages, which harms the performance of file-intensive applications; and it adopts a passive, reactive reclamation strategy that is triggered only when memory is under pressure, failing to fully utilize high-performance heterogeneous backends for active cold page offloading.
    
To address these limitations of traditional Linux memory management and fully leverage the advantages of heterogeneous storage technologies, this dissertation proposes an adaptive active cold page offloading solution for heterogeneous backends. This solution addresses two key challenges: determining appropriate offloading scale for different heterogeneous backends and accurately identifying hot and cold pages in the system. The core innovations of this approach are:

\begin{itemize}
    \item Design of an adaptive active offloading mechanism based on memory pressure, capable of adaptively determining the offloading scale for different heterogeneous backends. This mechanism uses memory synchronous reclamation delay time as a pressure indicator, enabling intelligent offloading adjustments across diverse application scenarios, and precisely controls the active offloading of application memory through integration with the Linux CGroup framework.

    \item Proposal of a hot-cold page identification algorithm based on reuse distance (the number of pages between two consecutive accesses to a page). Addressing the issue that existing page identification algorithms tend to reclaim file pages, which is unfriendly to file-intensive applications, this algorithm based on page reuse distance can more accurately identify hot file pages, thereby improving the performance of file-intensive applications.
\end{itemize}

A prototype system was developed and experimental validation for containerized application scenarios was conducted. The results demonstrate that the system can effectively identify container working sets in different heterogeneous backend storage architectures. While ensuring service performance, it achieves a significant reduction of 5\% to 14\% in container memory usage compared to baseline systems.


\englishkeyword{Tiered Memory, Working Set Size, Reuse Distance, Hot-Cold Page Identification}

\end{englishabstract}


