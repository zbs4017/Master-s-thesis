\begin{englishabstract}

With the rapid development of data-intensive applications such as machine learning, graph computing, and cloud computing in recent years, system memory capacity requirements have increased dramatically. The virtualization overhead introduced by containerized deployments has further exacerbated memory demands. Meanwhile, traditional DRAM technology faces physical scaling limitations and market price fluctuations, leading to continuously rising memory costs. Concurrently, application memory utilization remains generally low, with substantial amounts of cold memory. On the other hand, emerging storage and interconnection technologies—including NVMe SSDs, Non-Volatile Memory (NVM), Compute Express Link (CXL) devices, and RDMA-based remote memory access—are rapidly developing, creating new opportunities for memory system architecture. Software-based memory compression technologies continue to advance as well. These emerging technologies provide higher storage capacity while significantly reducing per-unit cost and power consumption.

Against this background, proactively offloading cold memory from applications to heterogeneous backends has become a promising cost optimization strategy. This strategy implements access to heterogeneous offload backends through the kernel's page fault mechanism, making the entire offloading process completely transparent to user programs without requiring any application modifications. Although high-performance heterogeneous backends can minimize the impact of offloading on application performance, this strategy still faces two key challenges: determining the optimal offloading scale and accurately identifying cold memory. Existing research is often limited to optimizations for specific offloading backends or application scenarios, lacking adaptability and struggling to achieve flexible adaptation across different offloading backends and application scenarios.
    
Addressing these issues, this thesis presents the following main contributions:
\begin{itemize}
    \item The research proposes an adaptive active offloading scheme based on memory pressure. This scheme uses the time occupied by memory synchronous reclamation as a basis for pressure measurement, enabling adaptive determination of offloading scale across heterogeneous backends and different application scenarios. Combined with CGroup's resource reclamation mechanism, it achieves precise offloading of application memory.
    \item Additionally, the research introduces a hot-cold page identification algorithm based on reuse distance (the number of pages between two consecutive accesses to a page). Targeting the problem that existing page hot-cold identification algorithms tend to reclaim file pages and are unfriendly to file-intensive applications, this method based on page reuse can more accurately identify hot file pages, thereby improving the performance of file-intensive applications.
\end{itemize}
    
A prototype system was developed and experimental validation for containerized application scenarios was conducted. The results demonstrate that the system can effectively identify container working sets in different heterogeneous backend storage architectures. While ensuring service performance, it achieves a significant reduction of 5\% to 14\% in container memory usage compared to baseline systems.


\englishkeyword{Hierarchical Memory, Working Set Size, Reuse Distance, Hot and Cold Page Identification}

\end{englishabstract}


